\documentclass[a4paper]{article}
\usepackage[utf8]{inputenc}
\usepackage{amsmath}
\usepackage{amssymb}
\usepackage{amsfonts}
\usepackage{tikz}
\usepackage{pgfplots}
\usepackage{subcaption}
\usepackage[backend=bibtex]{biblatex}
\usepackage[italian]{babel}


\numberwithin{equation}{section}
\numberwithin{figure}{section}

\addbibresource{documentazione.bib}
\newcommand{\aj}{The Astronomic Journal}
\newcommand{\hess}[1]{\mathbf{H}_{#1}}

\title{
	Studio analitico e numerico del sistema di Hénon-Heiles\\
	\vspace{1cm}	
	"Complementi di Matematica: EDO e le loro applicazioni"\\
	\begin{large}
		Milo Viviani
	\end{large}
	\vspace{1cm}

}

\date{25 Marzo 2025}
\author{Lorenzo Mugnaioli, Daniele Pisani}

\begin{document}
\maketitle
\clearpage
\tableofcontents
\clearpage

\section{Introduzione}
Nel 1962 nell'università di Princeton, Michael Hénon e  Carl Heiles,
stavano lavorando sul movimento delle stelle attorno a un centro galattico,
con l'ipotesi che il potenziale gravitazionale della galassia fosse tempo-invariante e simmetrico
rispetto ad un asse, ovvero, usando un sistema di coordinate cilindriche $(R,\theta,z)$ si avrebbe 
un potenziale in funzione solo di $R$ e $z$, $U(R,z)$.

In particolare si sono chiesti:
quale parte dello spazio delle fasi 6-dimesionale $(R,\theta,z, \dot{R},\dot{\theta}, \dot{z})$
viene riempito dalle orbite dei corpi su lunghi periodi di tempo? \cite{1964henonheiles}
Dove per lungi periodi si intende il tempo necessario al corpo per compiere molte rivoluzioni
intorno alla galassia.

A questa domanda hanno provato a dare risposta cercando di studiare la natura degli integrali del moto,
ovvero le funzioni il cui valore rimane invariato durante il moto del corpo (ad esempio l'energia,
che è proprio il primo integrale).

Ad ogni integrale corrisponde una ipersuperficie conetnuta nello spazio delle fasi; trattandosi di uno spazio
delle fasi a 6 dimensioni, sappiamo che esitono 5 integrali $I_i,\ 1\le i\le 5$ e le soluzioni
sono contenute nell'intersezione delle 5 ipersuperfici al variare delle costanti
$C_i,\ 1\le i\le 5$ che vengono imposte come condizioni iniziali
\begin{equation}
	I_i = C_i\ \ \  1,\le i\le 5
\end{equation}

Gli integrali possono essere di 2 tipi: isolanti o non-isolanti. Un integrale si dice non-isolante
se la sua ipersuperficie associata riempie lo spazio delle fasi in maniera densa, risulta quindi inutile
ai fini pratici per stabilire un confine per il sostegno della soluzione nel tempo.

Di norma si ignorano gli integrali non-isolanti e si fa riferimento agli integrali isolanti semplicemente
come \textit{integrali}.

Due integrali isolanti in questo caso sono noti, sono rispettivamente l'energia (\ref{eq:consenergia})
e il momento angolare (\ref{eq:consmomangolare}):
\begin{eqnarray}
	&&I_1 = U(R,z) + \frac{1}{2}(\dot{R}^2+R^2\dot{\theta}^2+\dot{z}^2) \label{eq:consenergia}\\
	&&I_2 = R^2\dot{\theta} \label{eq:consmomangolare}
\end{eqnarray}

È stato mostrato che $I_4$ ed $I_5$ sono in genere non-isolanti, ma la natura dell'integrale
$I_3$ non è nota.

Hénon e Heiles hanno quindi proceduto simulando il problema considerando un caso più generale
e trascurando la natura astronomica del problema, hanno cioè provato a studiare l'esistenza
di un terzo integrale per un generico potenziale con simmetria assiale $U(R,z)$.

\clearpage

\section{Riduzione al moto su un piano}
Il problema è riducibile al moto di un punto materiale su un piano con potenziale arbitrario.
Se consideriamo:
\begin{equation}
	U(R,z) = \bar{U}(R,z) + \frac{C_2}{2R^2}
\end{equation}
allora otteniamo
\begin{align}
	&\ddot{R} = -\frac{\partial U}{\partial R}&
	\ddot{z} = -\frac{\partial U}{\partial z}
\end{align}
e a questo punto è facile notare come questo problema sia del tutto equivalente a quello del moto su un piano
con potenziale arbitrario.

D'ora in avanti, lo spazio delle fasi sarà rappresentato da $(x,y,\dot{x},\dot{y})$,
con $x$ e $y$ che vanno a sostituire $R$ e $z$. Con questo spazio delle fasi si hanno 3 integrali,
e sappiamo che $I_1 = U(x,y)+\frac{1}{2}(\dot{x}^2+\dot{y}^2)$, l'energia totale, è isolante e $I_3$ è
in generale non isolante.

Non esiste un'integrale per il momento angolare dato che nel caso generale $U(x,y)$ non ha
simmetrie.

Si vuole quindi scoprire la natura del secondo integrale $I_2$.

Sapendo 3 delle 4 componenti dello spazio delle fasi e l'energia iniziale, ad esempio $(x,y,\dot{y})$,
è possibile ricavare la quarta, in questo caso $\dot{x}$, tramite:
\begin{equation}
	U(x,y)+\frac{1}{2}(\dot{x}^2+\dot{y}^2) = E \label{eq:consenergiaxy}
\end{equation}
da cui si ricava $\dot{x} = \pm \sqrt{2E-2U(x,y)-\dot{y}^2}$

Dato che $\dot{x}^2 \ge 0$ possiamo dire che
\begin{equation}
	U(x,y)+\frac{1}{2}\dot{y}^2 \le E
\end{equation}
e tale equazione definisce (solitamente) un volume limitato in $(x,y,\dot{y})$.

In assenza di un secondo integrale isolante, il volume viene eventualmente riempito
dagli stati assunti dal corpo con lo scorrere del tempo.
Altrimenti, il sistema composto da (\ref{eq:consenergiaxy}) e $I_2 = C_2$ forma una
superficie nella quale il moto del corpo rimane confinato.

Per controllare l'esistenza di un secondo integrale, vengono considerate le intersezioni della
traiettoria con un piano detto taglio di Poincaré. In questo caso, vengono considerati i punti
$(y,\dot{y})$ che soddisfano $x=0$ e $\dot{x}>0$ e si controlla che in un tempo sufficiente lungo
(idealmente infinito) i punti ottenuti non riempiano una superficie sul piano $(y,\dot{y})$. Nel
caso in cui ciò avvenga, la natura del secondo integrale sarà confermata essere non-isolante. Mentre
nel caso in cui tali punti rimangano confinati lungo delle curve, $I_2$ sarà confermato essere di natura
isolante.

\section{Potenziale di Hénon-Heiles}

%region figure potenziale
\begin{figure}[h!]
	\centering
	\begin{subfigure}[t]{.5\textwidth}
		\centering
		\includegraphics[width=\linewidth]{../img/potenziale_henon_heiles.png}
		\caption{Il potenziale di Hénon-Heiles}
		\label{img:potenziale3d}
	\end{subfigure}%
	\begin{subfigure}[t]{.5\textwidth}
		\centering
		\includegraphics[width=\linewidth]{../img/curve_livello_henon_heiles.png}
		\caption{In rosso, il punto di equilibrio stabile $(0,0)$ e le linee date da
		$U(q)=\frac{1}{6}$} 
		\label{img:curvelivello}
	\end{subfigure}
	\caption{Rappresentazioni grafiche del potenziale}
	\label{img:potenzialehh}
\end{figure}
%endregion

Prima di cominciare con la sperimentazione numerica, è necessario scegliere un potenziale e un
metodo numerico adeguati.

\begin{eqnarray}	
	&&U(q) = \frac{1}{2}(q_x^2+q_y^2+2q_x^2q_y-\frac{2}{3}q_y^3) \label{eq:potenziale}
\end{eqnarray}
Tale potenziale (\ref{img:potenzialehh}) è stato ricavato dopo alcuni tentativi
\cite{1964henonheiles}.
I motivi che hanno portato alla scelta della funzione (\ref{eq:potenziale}) sono: (1)
è semplice analiticamente e rende facile una computazione della traiettoria (si pensi al fatto
che l'articolo originale risale al 1963);
(2) presenta comunque soluzioni con proprietà non banali \cite{hairer}. 
Hénon e Heiles ritengono inoltre che il potenziale (\ref{eq:potenziale}) sia rappresentativo del caso
generico e che non ci siano cambiamenti fondamentali con l'inserimento di termini di ordine maggiore.

\subsection{Studio qualitativo del potenziale}

Iniziamo lo studio andando a calcolare gradiente ed hessiana di $U(q)$:
\begin{eqnarray}
	\nabla_q U(q)= 
	\begin{bmatrix}
		q_x + 2q_xq_y\\
		q_x^2 - q_y^2 + q_y
	\end{bmatrix}\\[0.2cm]
% 
	\hess{U}(q)=
	\begin{bmatrix}
		1 + 2q_y & 2q_x\\
		2q_x & 1 - 2q_y
	\end{bmatrix}
\end{eqnarray}

Risolvendo per $\nabla_q U = \mathbf{0}$ otteniamo 4 soluzioni, cioè punti di equilibrio
$\hat{q}_i = (q_{xi}, q_{yi})$:
\begin{eqnarray*}
	\hat{q}_1 = (0,0)\\
	\hat{q}_2 = (0,1)\\
	\hat{q}_3 = (-\frac{\sqrt{3}}{2}, -\frac{1}{2})\\
	\hat{q}_4 = (\frac{\sqrt{3}}{2}, -\frac{1}{2})
\end{eqnarray*}

Dallo studio dell'hessiana su questi 4 punti otteniamo:
\begin{eqnarray*}
\hess{U}(\hat{q}_1) =
	\begin{bmatrix}
		1& 0\\ 0 & 1
	\end{bmatrix} &definita\ positiva\\
%
\hess{U}(\hat{q}_2) = 
	\begin{bmatrix}
		3 & 0 \\ 0 & -1	
	\end{bmatrix} &indefinita\\
%
\hess{U}(\hat{q}_3) = 
	\begin{bmatrix}
		0 & -\sqrt{3} \\
		-\sqrt{3} & 2
	\end{bmatrix}&indefinita\\
%
\hess{U}(\hat{q}_4) = 
	\begin{bmatrix}
		0 & \sqrt{3} \\
		\sqrt{3} & 2
	\end{bmatrix}&indefinita\\
\end{eqnarray*}

$\hat{q}_1$ risulta essere l'unico punto di equilibrio stabile e gli altri sono punti di sella, in particolare,
$q_2, q_3, q_4$ sono proprio le intersezioni date dalle linee di $U(q)=\frac{1}{6}$, rappresentate in rosso nel
grafico \ref{img:curvelivello}. I punti $q_2, q_3, q_4$ formano un triangolo equilatero con centro $q_1$.

\subsection{Esistenza globale di una soluzione}
Dato che vale la legge di conservazione dell'energia e che l'energia cinetica non può essere negativa,
le soluzioni per $H \le \frac{1}{6}$ con posizione iniziale appartenente alla regione interna al triangolo
di vertici $q_2, q_3, q_4$ rimangono confinate al suo interno. Questo perché l'energia potenziale non può
superare l'energia totale. Per cui esistono soluzioni globali.

Studieremo quindi le orbite ottenute da $H(p,q) \le \frac{1}{6}$ e con posizione di partenza interna al triangolo

\section{Integrazione numerica}
Per i sistemi hamiltoniani, le equazioni del moto sono date da:
\begin{eqnarray}
	\dot{p} = -\frac{\partial H}{\partial q}\\
	\dot{q} = \frac{\partial H}{\partial p}
\end{eqnarray}

Esprimibile in forma più compatta
\begin{equation}
	\frac{d}{dt}
	\begin{bmatrix}
		p\\q
	\end{bmatrix} =
	\begin{bmatrix}
		\dot{p}\\\dot{q}
	\end{bmatrix} = 
	J^{-1}\nabla H(p,q)
\end{equation}
dove $J$ è la matrice simplettica fondamentale $2n\times 2n$ definita come $\begin{bmatrix}
O&I\\-I&O \end{bmatrix}$. La sua inversa è $\begin{bmatrix}
O&-I\\I&O \end{bmatrix}$.

Utilizzando $H(p,q) = U(q)+\frac{1}{2}(p_x^2+p_y^2)$, dove $U$ è il potenziale (\ref{eq:potenziale})
descritto nella sezione precedente, si ricavano le
seguenti equazioni:
$$
\begin{cases}
	\dot{q_x} = p_x\\
	\dot{q_y} = p_y\\
	\dot{p_x} = -q_x-2q_x q_y\\
	\dot{p_y} = -q_x^2 + q_y^2 - q_y
\end{cases}
$$
Ogni secondo membro di ciascuna di queste equazioni risulta differenziabile rispetto a
$(q_x,q_y,p_x,p_y)$ per ogni punto $\in \mathbb{R}^4$. Perciò sono funzioni localmente
Lipschitziane e le soluzioni del problema di Cauchy sono sempre uniche.

\subsection{Semplificazione di Størmer-Verlet per hamiltoniana separabile}
Come nella maggior parte dei problemi di meccanica classica, la funzione hamiltoniana
può essere espressa come somma di 2 funzioni:
\begin{equation}
	H(p,q) = T(p) + U(q)
\end{equation}
riconducibili rispettivamente all'energia cinetica e all'energia potenziale.

Proviamo ad usare questa proprietà per semplificare il metodo di Størmer-Verlet:
$$
\begin{cases}
	p_{n+\frac{1}{2}} = p_n - \frac{h}{2}\nabla_q H(p_{n+\frac{1}{2}},q_n)\\
	q_{n+1} = q_n + h \nabla_p H(p_{n+\frac{1}{2}}, q_n)\\
	p_{n+1} = p_{n+\frac{1}{2}} - \frac{h}{2} \nabla_q H(p_{n+\frac{1}{2}},q_{n+1})
\end{cases}
\implies
\begin{cases}
	p_{n+\frac{1}{2}} = p_n - \frac{h}{2}\nabla U(q_n)\\
	q_{n+1} = q_n + h \nabla T(p_{n+\frac{1}{2}})\\
	p_{n+1} = p_{n+\frac{1}{2}} - \frac{h}{2} \nabla U(q_{n+1})
\end{cases}
$$
e quindi:
$$
\begin{cases}
	q_{n+1} = q_n + h \nabla T(p_n - \frac{h}{2}\nabla U(q_n))\\
	p_{n+1} = p_{n+\frac{1}{2}} - \frac{h}{2} (\nabla U(q_n)+\nabla U(q_{n+1}))
\end{cases}
$$

Sostituendo con $\nabla T(p)=p$:
$$
\begin{cases}
	q_{n+1} = q_n + h(p_n-\frac{h}{2}\nabla U(q_n))\\
	p_{n+1} = p_n - \frac{h}{2} (\nabla U(q_n)+\nabla U(q_{n+1}))
\end{cases}
$$

Il risultato ottenuto è molto utile, in quanto non solo ci consente di calcolare il prossimo
stato della simulazione in solo 2 passaggi, ma rende Størmer-Verlet un metodo esplicito, nonostante
sia implicito nel caso generico.

Questo consente sia di velocizzare i calcoli sia di semplificare molto l'implementazione su MatLab, in
quanto non c'è bisogno di mettere in atto metodi per risolvere equazioni del tipo $x = F(x)$, come
Newton o iterazione di punto fisso.

\subsection{Risultati}
Utilizzeremo il metodo Størmer-Verlet come riferimento, dato che si comporta come la soluzione
esatta \cite{hairer} e lo confronteremo con Eulero implicito ed Eulero esplicito.

Data la tendenza di Eulero esplicito ad aumentare l'energia presente nel sistema e la tendenza
di Eulero implicito a diminiure l'energia, Eulero esplicito verrà usato per i casi con energia
bassa ($H=\frac{1}{12}, H=\frac{1}{10}$) menter Eulero implicito verrà usato nei sistemi con
energia alta ($H=\frac{1}{8}, H=\frac{1}{6}$)

Nei seguenti grafici mostreremo l'insieme delle intersezioni con il taglio di Poincaré (a colori
diversi corrispondono condizioni iniziali differenti) e l'andamento dell'Hamiltoniana per una delle
orbite rappresentate.

In Figura \ref{img:stormerverlet} si nota come l'hamiltoniana presenta un errore piccolissimo rispetto
al valore iniziale e segue un andamento periodico senza mai divergere, questo è il comportamento che 
ci si aspetta da un metodo numerico simplettico.

Dato che siamo sempre attorno al valore esatto per l'energia ma non c'è conservazione perfetta, si dice che
l'energia viene \textit{pseudoconservata} .

È possibile notare come all'aumentare dell'energia di partenza la regione che ammette un secondo
integrale sia sempre più ristretta.
\clearpage
\begin{figure}[h!]
	\centering
	\begin{subfigure}[t]{.49\textwidth}
		\centering
		\includegraphics[width=\linewidth]{../img/grafico_H_1-12_strVrl.png}
	\end{subfigure}
	\begin{subfigure}[t]{.49\textwidth}
		\centering
		\includegraphics[width=\linewidth]{../img/grafico_H_1-10_strVrl.png}
	\end{subfigure}
	\begin{subfigure}[t]{.49\textwidth}
		\centering
		\includegraphics[width=\linewidth]{../img/grafico_H_1-8_strVrl.png}
	\end{subfigure}
	\begin{subfigure}[t]{.49\textwidth}
		\centering
		\includegraphics[width=\linewidth]{../img/grafico_H_1-6_strVrl.png}
	\end{subfigure}

	\caption{Taglio di Poincaré e andamento dell'hamiltoniana per diversi valori di H,
	ottenuti dal metodo di Størmer-Verlet, 4000 intersezioni per orbita, h=0.01}
	\label{img:stormerverlet}
\end{figure}

Le simulazioni svolte con Eulero esplicito ed implicito risultano molto lontane da ciò che ci
si aspetta. Risulta molto inferiore anche il numero di intersezioni studiato.
Per Eulero esplicito infatti, il corpo esce dalla zona precentemente designata molto velocemente
e smette di intersecare il piano $q_x=0$ dopo relativamente poche rivoluzioni. D'altra parte,
l'enerigia in Eulero implicito segue un trend decrescente esponenziale e basta poco tempo affinché
il corpo si fermi al centro, rendendo superfluo uno studio più approfondito.
\clearpage
\begin{figure}[h!]
	\centering
	\begin{subfigure}[t]{.49\textwidth}
		\centering
		\includegraphics[width=\linewidth]{../img/grafico_H_1-12_expEuler.png}
	\end{subfigure}
	\begin{subfigure}[t]{.49\textwidth}
		\centering
		\includegraphics[width=\linewidth]{../img/grafico_H_1-10_expEuler.png}
	\end{subfigure}
	\begin{subfigure}[t]{.49\textwidth}
		\centering
		\includegraphics[width=\linewidth]{../img/grafico_H_1-8_implEuler.png}
	\end{subfigure}
	\begin{subfigure}[t]{.49\textwidth}
		\centering
		\includegraphics[width=\linewidth]{../img/grafico_H_1-6_implEuler.png}
	\end{subfigure}

	\caption{Taglio di Poincaré e andamento dell'hamiltoniana per diversi valori di H,
	400 intersezioni per orbita, h=0.0005}
\end{figure}

Un secondo tentativo è stato fatto con un passo di discretizzazione 10 volte più piccolo
(Figura \ref{img:euleroesplicitopiccolo}). Il risultato è notevolmente migliorato rispetto
alla sperimentazione precedente.

Nonostante ciò si nota che le orbite per le quali dovrebbe esistere un secondo integrale
isolante non formano linee pulite e l'errore introdotto dal metodo è chiaramente visibile,
soprattutto se si osserva il grafico dell'energia.

\begin{figure}[h!]
	\centering
	\begin{subfigure}[t]{.49\textwidth}
		\centering
		\includegraphics[width=\linewidth]{../img/grafico_H_1-12_expEuler_passoPiccolo.png}
	\end{subfigure}
	\begin{subfigure}[t]{.49\textwidth}
		\centering
		\includegraphics[width=\linewidth]{../img/grafico_H_1-10_expEuler_passoPiccolo.png}
	\end{subfigure}
	\begin{subfigure}[t]{.49\textwidth}
		\centering
		\includegraphics[width=\linewidth]{../img/grafico_H_1-8_implEuler_passoPiccolo.png}
	\end{subfigure}
	\begin{subfigure}[t]{.49\textwidth}
		\centering
		\includegraphics[width=\linewidth]{../img/grafico_H_1-6_implEuler_passoPiccolo.png}
	\end{subfigure}

	\caption{Taglio di Poincaré e andamento dell'hamiltoniana per diversi valori di H,
	ottenuti dal metodo di Eulero esplicito, 400 intersezioni per orbita, h=0.00005}
	\label{img:euleroesplicitopiccolo}
\end{figure}

\clearpage
\printbibliography
\end{document}
