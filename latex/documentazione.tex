\documentclass[a4paper, openright, twoside]{article}
\usepackage[utf8]{inputenc}
\usepackage{amsmath}
\usepackage{tikz}
\usepackage{pgfplots}




\renewcommand{\contentsname}{Indice}
\numberwithin{equation}{subsection}

\title{
	Integrazione numerica del sistema di Hénon-Heiles con metodi numerici simplettici\\
	\vspace{1cm}	
	Complementi di Matematica\\
	\begin{large}
	Milo viviani
	\end{large}

}

\date{25 Marzo 2025}
\author{Lorenzo Mugnaioli, Daniele Pisani}

\begin{document}
\maketitle
\clearpage
\tableofcontents
\clearpage

\section{Il sistema Hénon-Heiles}
\subsection{Cenni storici}

\subsection{Descrizione del modello}

Il modello di Hénon-Heiles è un sistema Hamiltoniano non lineare e non integrabile.
La funzione hamiltoniana è data dalle seguenti equazioni per l'energia cinetica e
per il potenziale (Figura \ref{img:potenzialehh}). Si noti che per convenienza, la massa viene normalizzata ($m=1$)
\begin{eqnarray}	
	&H(q,p) = T(p) + U(q) \label{eq:ham}\\
	&T(p) = \frac{1}{2}(p_x^2+p_y^2) \label{eq:cinetica}\\
	&U(q) = \frac{1}{2}(q_x^2+q_y^2+2q_x^2q_y-\frac{2}{3}q_y^3) \label{eq:potenziale}
\end{eqnarray}

Il potenziale può essere meglio compreso se espresso nella sua forma polare

\begin{equation}
	\frac{1}{2}r^2 + \frac{1}{3}r^3sin(3\theta)
\end{equation}

\begin{figure}[h!]
	\centering
	\includegraphics[width=10cm]{../img/potenziale_henon_heiles.png}
	\caption{Il potenziale di Hénon-Heiles}
	\label{img:potenzialehh}
\end{figure}

Come tutti i sistemi hamiltoniani, le equazioni del moto sono date da:

\begin{eqnarray}
	\dot{p} = -\frac{\partial H}{\partial q}\\
	\dot{q} = \frac{\partial H}{\partial p}
\end{eqnarray}
Esprimibile in forma più compatta

\begin{equation}
	\frac{d}{dt}
	\begin{bmatrix}
		p\\q
	\end{bmatrix} =
	\begin{bmatrix}
		\dot{p}\\\dot{q}
	\end{bmatrix} = 
	J^{-1}\nabla H(p,q)
\end{equation}

Dove $J$ è una matrice $2n\times 2n$ definita come $\begin{bmatrix}
	O&I\\-I&O \end{bmatrix}$. La sua inversa è $\begin{bmatrix}
	O&-I\\I&O \end{bmatrix}$


\subsection{Integrazione numerica}

\end{document}
