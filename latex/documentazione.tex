\documentclass[a4paper]{article}
\usepackage[utf8]{inputenc}
\usepackage{amsmath}
\usepackage{tikz}
\usepackage{pgfplots}
\usepackage{subcaption}
\usepackage[backend=bibtex]{biblatex}
\usepackage[italian]{babel}


\numberwithin{equation}{section}
\numberwithin{figure}{section}

\addbibresource{documentazione.bib}
\newcommand{\aj}{The Astronomic Journal}
\title{
	Studio analitico e numerico del sistema di Hénon-Heiles\\
	\vspace{1cm}	
	"Complementi di Matematica: EDO e le loro applicazioni"\\
	\begin{large}
		Milo Viviani
	\end{large}
	\vspace{1cm}

}

\date{25 Marzo 2025}
\author{Lorenzo Mugnaioli, Daniele Pisani}

\begin{document}
\maketitle
\clearpage
\tableofcontents
\clearpage

\section{Introduzione}
Nel 1962 all'università di Princeton, Michael Hénon e  Carl Heiles,
stavano lavorando sul movimento delle stelle attorno a un centro galattico.
Usando un sistema a coordinate cilindriche $(R,\theta,z)$, si sono chiesti:
quale parte dello spazio delle fasi 6-dimesionale $(R,\theta,z, \dot{R},\dot{\theta}, \dot{z})$
viene riempito dalle orbite su lunghi periodi di tempo?\cite{1964henonheiles}
Dove per molto tempo si intende molte rivoluzioni intorno alla galassia.

A questa domanda hanno provato a dare risposta cercando di trovare degli integrali isolanti,
ovvero delle funzioni il cui valore rimane invariato nel tempo.

Ad ogni integrale corrisponde un iperpiano sullo spazio delle fasi, trattandosi di uno spazio
delle fasi a 6 dimensioni, sappiamo che esitono 5 integrali $I_i,\ 1\le i\le 5$ e le soluzioni
sono rappresentate dall'intersezione delle 5 ipersuperfici, al variare delle costanti
$C_i,\ 1\le i\le 5$ che vengono imposte

\begin{equation}
	I_i = C_i,\ 1,\le i\le 5
\end{equation}

In particolare, per studiare quale parte dello spazio delle fasi venisse riempita, Hénon e Heiles
si sono concentrati sugli integrali isolanti, dato che gli integrali non-isolanti hanno come
ipersuperficie un'infinità di fogli che riempie lo spazio delle fasi in maniera
densa\cite{1964henonheiles}.

Gli integrali non-isolanti risultano quindi inutili nella pratica, vengono quindi ignorati e
si fa riferimento agli integrali isolanti semplicemente come integrali.

2 integrali isolanti in questo caso sono noti, sono rispettivamente l'energia (\ref{eq:consenergia})
e il momento angolare (\ref{eq:consmomangolare})

\begin{eqnarray}
	&&I_1 = U(R,z) + \frac{1}{2}(\dot{R}^2+R^2\dot{\theta}^2+\dot{z}^2) \label{eq:consenergia}\\
	&&I_2 = R^2\dot{\theta} \label{eq:consmomangolare}
\end{eqnarray}

In generale, è stato mostrato che $I_4$ ed $I_5$ sono non-isolanti, ma la natura dell'integrale
$I_3$ non è nota.

Hénon e Heiles hanno quindi proceduto simulando il problema considerando un caso più generale
e dimenticando la natura astronomica del problema, hanno cioè provato a studiare l'esistenza
di un terzo integrale per un generico potenziale con simmetria rotazionale $U(R,z)$ generico.

\clearpage
\section{Riduzione al moto su un piano}
Il problema è del tutto equivalente al moto di un punto materiale su un piano con potenziale
arbitrario. D'ora in avanti, lo spazio delle fasi sarà rappresentato da $(x,y,\dot{x},\dot{y})$,
con $x$ e $y$ che vanno a sostituire $R$ e $z$. Con uno spazio delle fasi si hanno 3 integrali,
e sappiamo che $I_1 = U(x,y)+\frac{1}{2}(\dot{x}^2+\dot{y}^2)$, l'energia totale, è isolante e $I_3$ è
in generale non isolante.

Non esiste un'integrale per il momento angolare dato che nel caso generale $U(x,y)$ non ha
simmetrie.

A questo punto si vuole scoprire la natura del secondo integrale $I_2$.

Sapendo 3 delle 4 componenti dello spazio delle fasi e l'energia iniziale, ad esempio $(x,y,\dot{y})$,
è possibile
ricavare la quarta, in questo caso $\dot{x}$ tramite
\begin{equation}
	U(x,y)+\frac{1}{2}(\dot{x}^2+\dot{y}^2) = E \label{eq:consenergiaxy}
\end{equation}
e in particolare $\dot{x} = \sqrt{2E-2U(x,y)-\dot{y}^2}$

Dato che $\dot{x}^2\ge0$ possiamo dire che
\begin{equation}
	U(x,y)+\frac{1}{2}\dot{y}^2 \le E
\end{equation}
e tale equazione definisce un volume chiuso in $(x,y,\dot{y})$.

In assenza di un secondo integrale isolante, tutto il volume viene eventualmente riempito,
altrimenti, il moto rimane confinato sulla superficie risultante dalle soluzioni del sistema
composto da (\ref{eq:consenergiaxy}) e $I_2 = C_2$.

Per controllare l'esistenza di un secondo integrale, vengono considerate le intersezioni della
traiettoria con un piano detto taglio di Poincaré. In questo caso, vengono considerati i punti
$(y,\dot{y})$ che soddisfano $x=0$ e $\dot{x}>0$ e si controlla che in un tempo sufficiente lungo
(idealmente infinito) i punti ottenuti non riempiano una superficie sul piano $(y,\dot{y})$. Nel
caso ciò avvenga, la natura del secondo integrale sarà confermata essere non-isolante, nel caso
invece tali punti rimangano confinati lungo delle curve, $I_2$ è di natura isolante.

\section{Potenziale di Hénon-Heiles}

%region figure potenziale
\begin{figure}[h!]
	\centering
	\begin{subfigure}[t]{.5\textwidth}
		\centering
		\includegraphics[width=\linewidth]{../img/potenziale_henon_heiles.png}
		\caption{Il potenziale di Hénon-Heiles}
		\label{img:potenziale3d}
	\end{subfigure}%
	\begin{subfigure}[t]{.5\textwidth}
		\centering
		\includegraphics[width=\linewidth]{../img/curve_livello_henon_heiles.png}
		\caption{In rosso, il punto di equilibrio stabile $(0,0)$ e le linee date da
		$U(q)=\frac{1}{6}$} 
		\label{img:curvelivello}
	\end{subfigure}
	\caption{Rappresentazioni grafiche del potenziale}
	\label{img:potenzialehh}
\end{figure}
%endregion

Prima di cominciare con la sperimentazione numerica, è necessario scegliere un potenziale e un
metodo numerico adeguati.

\begin{eqnarray}	
	&&U(q) = \frac{1}{2}(q_x^2+q_y^2+2q_x^2q_y-\frac{2}{3}q_y^3) \label{eq:potenziale}
\end{eqnarray}
tale potenziale (Figura \ref{img:potenzialehh}) è stato ricavato dopo alcuni tentativi
\cite{1964henonheiles}.
I motivi che hanno portato alla scelta funzione (\ref{eq:potenziale}) sono (1)
è semplice analiticamente e rende facile una computazione della traiettoria (si pensi al fatto
che l'articolo originale, \cite{1964henonheiles} in bibliografia, risale al 1963);
(2) presenta soluzioni con proprietà non banali\cite{hairer}. 
Hénon e Heiles ritengono anche che il potenziale (\ref{eq:potenziale}) sia rappresentativo del caso
generico e che non ci siano cambiamenti fondamentali con l'inserimento di termini di ordine maggiore.

\section{Integrazione numerica}

Come per tutti i sistemi hamiltoniani, le equazioni del moto sono date da:
\begin{eqnarray}
	\dot{p} = -\frac{\partial H}{\partial q}\\
	\dot{q} = \frac{\partial H}{\partial p}
\end{eqnarray}

Esprimibile in forma più compatta
\begin{equation}
	\frac{d}{dt}
	\begin{bmatrix}
		p\\q
	\end{bmatrix} =
	\begin{bmatrix}
		\dot{p}\\\dot{q}
	\end{bmatrix} = 
	J^{-1}\nabla H(p,q)
\end{equation}
dove $J$ è una matrice $2n\times 2n$ definita come $\begin{bmatrix}
O&I\\-I&O \end{bmatrix}$. La sua inversa è $\begin{bmatrix}
O&-I\\I&O \end{bmatrix}$.

Utilizzando $H(p,q) = U(q)+\frac{1}{2}(p_x^2+P_y^2)$, dove $U$ è il potenziale (\ref{eq:potenziale})
descritto nella sezione precedente, si ricavano le
seguenti equazioni:
$$
\begin{cases}
	\dot{q_x} = p_x\\
	\dot{q_y} = p_y\\
	\dot{p_x} = -q_x-2q_x q_y\\
	\dot{p_y} = -q_x^2 + q_y^2 - q_y
\end{cases}
$$

Usando il metodo Størmer-Verlet come riferimento, dato che si comporta come la soluzione
esatta\cite{hairer}, analizzeremo dal punto di vista della correttezza i metodi Eulero esplicito
ed Eulero implicito.

Nei seguenti grafici mostreremo l'insieme delle intersezioni con il taglio di Poincaré (a colori
diversi corrispondono condizioni iniziali differenti) e l'andamento dell'Hamiltoniana per una delle
orbite rappresentate.

\begin{figure}[h!]
	\centering
	\begin{subfigure}[t]{.49\textwidth}
		\centering
		\includegraphics[width=\linewidth]{../img/grafico_H_1-12_strVrl.png}
	\end{subfigure}
	\begin{subfigure}[t]{.49\textwidth}
		\centering
		\includegraphics[width=\linewidth]{../img/grafico_H_1-10_strVrl.png}
	\end{subfigure}
	\begin{subfigure}[t]{.49\textwidth}
		\centering
		\includegraphics[width=\linewidth]{../img/grafico_H_1-8_strVrl.png}
	\end{subfigure}
	\begin{subfigure}[t]{.49\textwidth}
		\centering
		\includegraphics[width=\linewidth]{../img/grafico_H_1-6_strVrl.png}
	\end{subfigure}

	\caption{Taglio di Poincaré e andamento dell'hamiltoniana per diversi valori di H,
	ottenuti dal metodo di Størmer-Verlet}
\end{figure}



\clearpage
\printbibliography
\end{document}
